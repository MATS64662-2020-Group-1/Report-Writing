\documentclass[\report.tex]{subfiles}

\begin{document}
\section{Introduction}
Finite element modelling (FEM) is the process of segmenting a defined geometry by meshing, and solving coupled partial differential physics equations for individual elements within the built mesh, given inputs of boundary conditions surrounding the geometry, a series of timesteps over which the simulation runs, and materials parameters that describe how the geometry will behave under the specified conditions. The defined geometry may be a two dimensional image or a three dimensional object. For the purposes of this project, microstructural images have been used to estimate the physical behavior of a material on the microscopic scale. Image processing software based in python is used to prepare such images for conversion to a finite element modelling geometry input. There are a variety of different open-source finite element solvers and image processing software available for use in this task, including: sfepy, OOF2, DAMASK, gmsh, sci-kit image and scipy.

\section{Aims and Objectives}
\subsection{Aims}

\begin{itemize}
  \item Pre-process images.
  \item Perform finite element analysis on pre-processed images.
  \item Discuss results.
\end{itemize}

\subsection{Objectives}

\begin{itemize}
  \item To pre-process several microstructural images through the use of sci-kit image and scipy in jupyter notebook by binarising the image into two distinct phases.
  \item To create a finite element mesh for the processed image using an adaptive algorithm.
  \item To research materials parameters for each of the respective phases present in the geometry.
  \item To define boundary conditions, field equations, material parameters and outputs of the simulation.
  \item To post-process outputs of the simulation in order to extract useful results, plot, and produce conclusions based on a comparison of simulated results to that of experimental results found in literature, with a consideration of the assumptions used to create the model.
  \item To describe possible future work to build upon the developed workflow.
\end{itemize}

\end{document}
