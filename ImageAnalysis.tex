\documentclass[\report.tex]{subfiles}

\begin{document}
\section{Image Analysis}
\noindent A key requirement for this project is to build a python script which will allow JPG images to be read as an array, which can then be manipulated and processed with python packages such as Sci-Kit Image (skimage) \cite{scikitimage}. Several microstructure images have been used in this project, from a variety of sources including previous research data, scientific literature and the provided database, all with full and sufficient permissions. Images were then pre-processed accordingly through various techniques such as segmentation, removal of noise and measurement, before being imported into FE simulations to be used to create a mesh. This section will further detail how this process was undertaken and broken down through the use of test cases to ensure that the python script and skimage were performing accurately and as desired.\\

\noindent The first test case that was carried out was processing an image of a ferritic microstructure taken from the Materials Science and Engineering: An Introduction by Callister \cite{CallisterJrWilliamD2000MSaE}. Firstly, an understanding of how to navigate and operate the skimage package was developed, as the software was new to the authors, through testing different algorithms and operations independently on the test image. This included manipulating image colour and contrast, converting the image to grayscale and cropping the image to remove the scale bar. Other features that were tested at this stage was the use of filters such as the Sobel filter, which is commonly used in image processing to emphasise the edges in an image, aiding edge detection algorithms. A watershed transform was another image processing technique that was trialled at this stage, where the image is treated as a topographical map, with brightness determining elevation of each point and then identifying the lines which run across the peaks to segment the map. Finally, these operations were combined to process the image as effectively as possible, before the measuring tools were used to estimate average grain size.\\

%%%%%% Insert images %%%%%%%%%%%%%%%%%%5

\noindent Secondly, in test case 2, the effectiveness of the code developed in the first test case was applied to an image of a thermally etched steel microstructure showing prior austenite grains, taken from previous research by Joshua Collins and authorised by their academic supervisor, Dr Ed Pickering. This microstructure image, as seen below, had previously been the subject of average grain size analysis via the linear intercept method. The average grain size would now be calculated the image processing python script and the two results compared to assess the accuracy of the approach. The image was processed using many of the techniques tested out previously, with the image first being read before being cropped to remove unfocused areas of the micrograph. The image was then segmented via the application of the Sobel filter, application of markers and then a watershed transform to fill in regions of the elevation map before the grains could finally be segmented and labelled individually. Finally, the image is converted back to gray scale, i.e. a binary image, to allow each grain to be assigned a phase number associated with material properties. This will allow the "image" data to be exported as a numpy array and saved as a Comma Seperated Value (CSV) file for easy importation into the Gmsh meshing software later.\\

\noindent As mentioned previously the successfulness of the second test case was assessed by comparing the calculated average grain size from the python algorithm in comparison to that previously measured using the linear intercept method. To do this, the area of each segmented grain size was measured and then the average diameter of each grain can be estimated, by assuming the grains are perfect circles. Although this is a large assumption which is clearly not true, it yielded excellent results as an average grain size of 23.5 microns was estimated, with the previously measured average of 24.9 microns. Such a small difference in average grain size clearly demonstrates the viability of this approach and provides reassurance that the algorithm should be applied to other images to further test the virtue of it.\\

%%%%%%%%%% Insert image of microstructure %%%%%%%%%%%%%%%555

\noindent Test case 3 was then commenced, seeking to apply the algorithm to another image, this time a brightfield image of an $\alpha$ + $\beta$ microstructure of the Ti-6Al-4V alloy, provided by Dr Pratheek Shantraj and shown below. The same procedure as in the previous test case was applied to the image and the image was succesffuly converted into a segmented, binary image and then exported into the Gmsh software in order to create a square based mesh for subsequent FEM analysis.
%%%%%%%%%%%%%%%% Images of Ti64 microstructure for test case 3 %%%%%%%%%%

\noindent Test case 4 then took the same image from the previous test case and processsed and segemnted a single grain for meshing and FEM analysis. This analysis was done with the use of the additional skimage sub-packages; threshold filters and clean border segmentation to remove the surrounding grains from the image. The binary, single grain image seen below demonstrates that this approach is valid for application to and isolation of individual grains, aswell as a multitude of grains as demonstrated in test case 3. The successful application of the 4 test cases allowed the authors to gain a high level of confidence in the algorithm and therefore move on to the next stage of the project, meshing and FE.
\end{document}
