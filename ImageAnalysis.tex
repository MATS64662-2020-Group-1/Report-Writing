\documentclass[\report.tex]{subfiles}

\begin{document}
\section{Image Analysis}
\noindent A key requirement for this project is to build a python script which will allow PNG images to be converted to data, which can then be manipulated and processed with python packages such as Sci-Kit Image (skimage) \cite{scikitimage}. Several microstructure images have been used in this project, from a variety of sources including previous research data, scientific literature and the provided database, all with full and sufficient permissions. Images were then pre-processed accordingly through various techniques such as segmentation, removal of noise and measurement, before being imported into FE simulations to be used to create a mesh. This section will further detail how this process was undertaken and broken down through the use of test cases to ensure that the python script and skimage were performing accurately and as desired.\\

\\\noindent The first test case that was carried out was processing an image of a ferritic microstructure taken from the Materials Science and Engineering: An Introduction by Callister \cite{CallisterJrWilliamD2000MSaE}. Firstly, an understanding of how to navigate and operate the skimage package was developed, as the software was new to the authors, through testing different algorithms and operations independently on the test image. This included manipulating image colour and contrast, converting the image to grayscale and cropping the image to remove the scale bar. Other features that were tested at this stage was the use of filters such as the Sobel filter, which is commonly used in image processing to emphasise the edges in an image, aiding edge detection algorithms. A watershed transform was another image processing technique that was trialled at this stage, where the image is treated as a topographical map, with brightness determining elevation of each point and then identifying the lines which run across the peaks. Finally, these operations were combined to process the image as effectively as possible, before the mesauring tools were used to estimate average grain size.

\\\noindent 
\end{document}
