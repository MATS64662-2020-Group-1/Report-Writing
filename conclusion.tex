\documentclass[report.tex]{subfiles}

\begin{document}

\section{Conclusion}

\noindent This project has successfuly obtained microstructural images that have been pre-processed 
 and imported into a simulation using available open-source FE and image-analysis packages. In terms of the image 
 preprocessing, multiple test cases were conducted until a function which is capable of segmenting and binarising a image
 is obtained. The function was subsequently used on sections of a globular Ti64 microstructure to be imported in the FE
 processing package: OOF2. OOF2 is based on a set of C++ classes for finite elements and material properties, 
 tied together in a Python infrastructure; hence, any simulation was adjusted with a few lines of Python code. 
 To ensure OOF2 was utilised correctly and to validate results, multiple tests 
 cases were also conducted for the FE processing side of the project. Each subsequent test case was increased in 
 model complexity, where the final simulation involved a combination of post-processing techniques from each test case, 
 the correct material parameters for the $\alpha$ and $\beta$ phases of Ti64, a refined FE mesh and defined boundary conditions and 
 field equations. The results of the final simulation inclueded contour maps for the temperature distribution,
 the displacements in the x and y directions and the stress concentrated areas in the microstructure. Although the aim of this 
 project was achieved, the results of the simulation have not been compared to any existing simulation results and the FE
 software package, OOF2, contains many bugs which make each simulation unreliable and unreproducible. \newline
 
\noindent Further work for this project could be to compare the globular microstructure simulation results to the lamellar 
microstructure for Ti64 and to analyse any noteable differences. An additional test case for the FE processing
could be introduced which includes a 3D model, which could be obtained from simply extruding 2D images. This can then 
be inputted into the FE package: OOF3D, which works in the same manner as OOF2 but in three dimensions. Although the 
homogeneity index was high, a mixture of element types (quad and tri) could also be used in the mesh to provide a 
more accurate representation of the microstructure. The image processing function can also be optimised through the use of
loops and combining scripts. Finally, a software package with proven reliability and less bugs could be used to provide 
reliable results e.g. SfePy, FiPy.
 
